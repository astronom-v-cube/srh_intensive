\documentclass[a4paper, 14pt]{extarticle}

% Шрифты, кодировки, символьные таблицы, переносы
\usepackage{cmap}
\usepackage[T2A]{fontenc}
\usepackage[utf8x]{inputenc}
\usepackage[english, russian]{babel}

% Пакеты американского математического сообщества
\usepackage{amssymb, amsfonts, amsmath, amsthm}  

%Библиография
\usepackage[style=authoryear-icomp,sorting=anyt]{biblatex}

% Сокращения
\usepackage{cancel}

\theoremstyle{definition}
\newtheorem{definition}{Определение}

% Красная строка
\usepackage{indentfirst}

% Ссылки в pdf
\usepackage[unicode, colorlinks, urlcolor=magenta, linkcolor=black]{hyperref}

% Таблицы
\usepackage{makecell,multirow} 

% Графика
\usepackage{graphicx}
\usepackage[usenames, dvipsnames]{color} 
\usepackage{float}
% \usepackage{subcaption}

% Геометрия страницы
\usepackage{geometry}
\geometry{left=2cm, right=2cm, top=2.5cm, bottom=2.5cm, bindingoffset=0cm, headheight=18pt}

% Колонтитулы
\usepackage{fancyhdr} 
% применим колонтитул к стилю страницы
\pagestyle{fancy} 
%очистим "шапку" страницы
\fancyhead{} 
%слева сверху на четных и справа на нечетных
\fancyhead[R]{Смирнов Д.А.} 
%справа сверху на четных и слева на нечетных
\fancyhead[L]{Интенсив на СРГ} 
%очистим "подвал" страницы
\fancyfoot{} 
% номер страницы в нижнем колинтуле в центре
\fancyfoot[C]{\thepage} 

% Межстрочный отступ
\usepackage{setspace}
\linespread{1.15} % капельку увеличенный
\frenchspacing % <<французские>> пробелы

% Нумерация
\renewcommand{\labelenumii}{\theenumii)}
% В заголовках появляется точка, но при ссылке на них ее нет
\usepackage{misccorr}

% Содержание
\usepackage{tocloft}
\usepackage{secdot}
\sectiondot{subsection}

% Физика
\usepackage{physics}

% Новые команды
\newcommand{\Mean}[1]{\langle#1\rangle}
\newcommand{\Defi}{\underset{def}{=}}
\newcommand{\Inte}{\int\limits_{-\infty}^{\infty}} 

\addto\captionsrussian{%
	\renewcommand{\contentsname}{Оглавление}
	\renewcommand{\partname}{Часть}%
}
\def\thepart{\Roman{part}}
\usepackage{tocloft}
\renewcommand{\cftpartleader}{\cftdotfill{\cftdotsep}} % for parts
\renewcommand{\cftpartpresnum}{\Roman{part}~} % Префикс для номеров частей
% \renewcommand{\cftchapleader}{\cftdotfill{\cftdotsep}} % for chapters
\renewcommand{\cftsecleader}{\cftdotfill{\cftdotsep}} % for chapters
% \newlength\mylen
\renewcommand\thepart{\Roman{part}.}
% \renewcommand\cftpartpresnum{Лекция~}
% \renewcommand\cftsecpresnum{Лекция~}
% \setlength{\cftsecnumwidth}{6em}
% \renewcommand{\cftsecpresnum}{Лекция\ }
% \renewcommand{\cftsecaftersnum}{.}
% \renewcommand{\cftsecaftersnumb}{\newline}
\renewcommand{\cftsecdotsep}{\cftdotsep}
\renewcommand{\kappa}{\varkappa}
\renewcommand{\phi}{\varphi}
\renewcommand{\epsilon}{\varepsilon}

% #1: math symbol
% #2: legend
\def\alegend#1#2{\overset{\underset{\scriptstyle\downarrow}{\scriptstyle\text{#2}}}{#1}}
\def\blegend#1#2{\underset{\underset{\scriptstyle\text{#2}}{\scriptstyle\uparrow}}{#1}}
\def\hp{\hat{p}}
\def\hx{\hat{x}}
\def\hH{\hat{H}}

% \usepackage[explicit]{titlesec}
% \titleformat{\section}{\normalfont\Large\bfseries}{}{0em}{Лекция\ \thesection.\ #1}
\usepackage{epigraph}
\usepackage{mathtools}
\mathtoolsset{showonlyrefs=true}

% https://tex.stackexchange.com/questions/8720/overbrace-underbrace-but-with-an-arrow-instead

\usepackage{xparse}% http://ctan.org/pkg/xparse

% Настройка формата ссылок на рисунки
\usepackage{cleveref}
\crefname{figure}{рис.}{рис.}

\newcommand{\pvec}[1]{\vec{#1}\mkern2mu\vphantom{#1}}
% Нормальный вектор для штрихов

\newcommand\undernote[2]{
	%
	\underarrow[
	#1
	][\uparrow]{\substack{#2}}
	%
}
\DeclareMathOperator{\Div}{div}
\DeclareMathOperator{\Rot}{rot}
\DeclareMathOperator{\Grad}{grad}

\usepackage{listings}

% Default fixed font does not support bold face
\DeclareFixedFont{\ttb}{T1}{txtt}{bx}{n}{12} % for bold
\DeclareFixedFont{\ttm}{T1}{txtt}{m}{n}{12}  % for normal

% Custom colors
\usepackage{color}
\definecolor{deepblue}{rgb}{0,0,0.5}
\definecolor{deepred}{rgb}{0.6,0,0}
\definecolor{deepgreen}{rgb}{0,0.5,0}

% Python style for highlighting
\newcommand\pythonstyle{\lstset{
		language=Python,
		basicstyle=\ttm,
		morekeywords={self},              % Add keywords here
		keywordstyle=\ttb\color{deepblue},
		emph={MyClass,__init__},          % Custom highlighting
		emphstyle=\ttb\color{deepred},    % Custom highlighting style
		stringstyle=\color{deepgreen},
		frame=tb,                         % Any extra options here
		showstringspaces=false
}}

% Python environment
\lstnewenvironment{python}[1][]
{
	\pythonstyle
	\lstset{#1}
}
{}

% Python for external files
\newcommand\pythonexternal[2][]{{
		\pythonstyle
		\lstinputlisting[#1]{#2}}}

% Python for inline
\newcommand\pythoninline[1]{{\pythonstyle\lstinline!#1!}}