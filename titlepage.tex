%!TEX root = lectionsA4.tex

% \newpage
\begin{titlepage}
% \thispagestyle{empty}

% \begin{flushleft}
% УДК 532.59(075.8) \\
% ББК 22.251я73\\

% \end{flushleft}
 
% \vskip 20pt
\begin{center}
Министерство науки и высшего образования Российской Федерации
\end{center}
\vskip 20pt
\begin{center}
Федеральное государственное автономное образовательное\\
учреждение высшего образования 
<<Национальный исследовательский Нижегородский государственный университет им. Н.И. Лобачевского>>\\[10pt]
Радиофизический факультет
\end{center}
\vskip 100pt
\begin{center}
	{Автор}
	\vspace{7pt}
\end{center}
\begin{center}
	{\bf \Large НАЗВАНИЕ}
\end{center}
\vskip 20pt
\begin{center}
	\it Учебное пособие
\end{center}
\vskip 20pt
\begin{center}
\fontsize{13pt}{1em}\selectfont
Рекомендовано для студентов
% Оформление и вёрстка: \\ это я
\end{center}

\vfill
\begin{center}
	Нижний Новгород\\
	\currentyear
\end{center}
\vspace{40pt}

\end{titlepage}
\begin{titlepage}
\begin{spacing}{0.92}
\thispagestyle{empty}
% \begin{flushleft}
\noindent\makebox[1.5cm]{УДК} номер номер номер\\ 
% \indent УДК 532, 533.6, 534.2\\ 
% \indent ББК 22.553\\
\noindent\makebox[1.5cm]{ББК}номер\
\noindent\makebox[1.5cm]{}
% \end{flushleft}
\vskip 1em
\begin{center}
	\textit{Рецензенты:}\\
	\textbf{умный человек}~--- д.ф.-м.н., профессор,\\
	\textbf{еще один умный человек}~--- д.ф.-м.н., профессор
\end{center}
\vskip 10pt
\noindent\makebox[1.5cm]{}\textbf{автор авторович}\\
\noindent\makebox[1.5cm]{Г--95}\hspace{1cm}\textbf{название}: учебное пособие /
\begin{adjustwidth}{1.5cm}{0cm}
Автор авторович~--- Нижний Новгород: Изд-во ННГУ им. Н.И. Лобачевского, 2024.~-- много~c.
\end{adjustwidth}
\vskip 10pt
\hspace{2.5cm}ISBN номер
\vskip 10pt
% \fontsize{12pt}{1em}\selectfont

\begin{adjustwidth}{1.5cm}{0cm}
\hspace{1cm}Учебное пособие такое то

% \noindent\hspace{1cm}Публикация данного пособия осуществлена при финансовой поддержке
% Министерства образования и науки Российской Федерации.

\end{adjustwidth}

% Учебное пособие рекомендовано Учёным советом радиофизического факультета
% для студентов ННГУ, обучающихся по направлениям подготовки 03.03.03 и 03.04.03
% ``Радиофизика'' (бакалавриат и магистратура), 02.03.02 ``Фундаментальная информатика и
% информационные технологии'' (бакалавриат), а также специальности 10.05.02
% ``Информационная безопасность телекоммуникационных систем''.

\vskip 1em
\begin{adjustwidth}{1.5cm}{0cm}
\begin{center}
\textit{Ответственный за выпуск:}\\
% заместитель председателя методической комиссии радиофизического факультета ННГУ, 
д.ф.-м.н., профессор умный человек
\end{center}
\vskip 3em
\end{adjustwidth}
% \vfill
\vskip 9em
ISBN номер \hfill УДК номер номер\\
\hphantom{a}\hfill ББК номер
\vskip 2em

\begin{adjustwidth}{6.5cm}{0cm}
\makebox[1cm]{\noindent\tikz[baseline=-0.3em]{\node[left, scale=0.7] (0,0) {\copyright}}}Нижегородский государственный\\
\makebox[1cm]{}университет им. Н.И. Лобачевского, 2024\\
\makebox[1cm]{\noindent\tikz[baseline=-0.3em]{\node[left, scale=0.7] (0,0) {\copyright}}} автор авторович, 2024
\end{adjustwidth}
\end{spacing}
% \end{flushright}

\end{titlepage}